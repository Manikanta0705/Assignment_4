\let\negmedspace\undefined
\let\negthickspace\undefined
%\RequirePackage{amsmath}
\documentclass[journal,12pt,twocolumn]{IEEEtran}
 \usepackage[utf8]{inputenc}
 \usepackage{graphicx}
 \usepackage{amsmath}
 \usepackage{mathrsfs}
\usepackage{txfonts}
\usepackage{stfloats}
\usepackage{bm}
\usepackage{cite}
\usepackage{cases}
\usepackage{subfig}
 \usepackage{amsfonts}
 \usepackage{amssymb}
 \usepackage{enumitem}
\usepackage{mathtools}
\usepackage{tikz}
\usepackage{circuitikz}
\usepackage{verbatim}
\usepackage[breaklinks=false,hidelinks]{hyperref}
\usepackage{listings}
\usepackage{calc}
\usepackage{float}
\usepackage{longtable}
\usepackage{multirow}
\usepackage{multicol}
\usepackage{color}
\usepackage{array}
\usepackage{hhline}
\usepackage{ifthen}
\usepackage{chngcntr}

\newcommand{\BEQA}{\begin{eqnarray}}
\newcommand{\EEQA}{\end{eqnarray}}
\newcommand{\define}{\stackrel{\triangle}{=}}
\bibliographystyle{IEEEtran}
%\bibliographystyle{ieeetr}
\def\inputGnumericTable{}
\let\vec\mathbf
\providecommand{\pr}[1]{\ensuremath{\Pr\left(#1\right)}}
\providecommand{\sbrak}[1]{\ensuremath{{}\left[#1\right]}}
\providecommand{\lsbrak}[1]{\ensuremath{{}\left[#1\right.}}
\providecommand{\rsbrak}[1]{\ensuremath{{}\left.#1\right]}}
\providecommand{\brak}[1]{\ensuremath{\left(#1\right)}}
\providecommand{\lbrak}[1]{\ensuremath{\left(#1\right.}}
\providecommand{\rbrak}[1]{\ensuremath{\left.#1\right)}}
\providecommand{\cbrak}[1]{\ensuremath{\left\{#1\right\}}}
\providecommand{\lcbrak}[1]{\ensuremath{\left\{#1\right.}}
\providecommand{\rcbrak}[1]{\ensuremath{\left.#1\right\}}}
%\providecommand{\abs}[1]{\left\vert#1\right\vert}
\providecommand{\res}[1]{\Res\displaylimits_{#1}}
\newcommand{\myvec}[1]{\ensuremath{\begin{pmatrix}#1\end{pmatrix}}}
\newcommand{\mydet}[1]{\ensuremath{\begin{vmatrix}#1\end{vmatrix}}}
\newcommand{\PROBLEM}{\noindent \textbf{PROBLEM: }}
\newcommand{\solution}{\noindent \textbf{Solution: }}
\newcommand{\note}{\noindent \textbf{Note: }}
\newcommand{\tofind}{\noindent \textbf{To find: }}
\title{Assignment 4}
\author{MANIKANTA UPPULAPU\\BT21BTECH11005}
\date{}
\begin{document}
% make the title area
\maketitle
\PROBLEM Find the sample space associated with the experiment of rolling a pair of dice (one is blue and the other red) once. Also, find the number of elements of this sample space.\\

\solution \\
Given, a pair of dice( one is blue and the other red).\\

\tofind sample space of the experiment and number of elements in the sample space.\\

let us denote the events of rolling one dice and two dice by random variables $X$ and $Y$ respectively.\\
Sample space of $X$ = $S_x$,where\\
\begin{enumerate}[label=(\roman{enumi})]
	$S_x = \cbrak{1,2,3,4,5,6}$\\
\end{enumerate}
The sample space for rolling two dice is,\\
\begin{enumerate}[label=(\roman{enumi})]
	$S_y = S_x \times S_x \text{(CARTESIAN PRODUCT)}$
\end{enumerate}
So,the sample for rolling two dice is given by\\
\begin{enumerate}[label=(\roman{enumi})]
$S_y = \cbrak {(x,y) : x \text{ is number on blue die and y is number on blue die} }$\\
\end{enumerate}
And $S_y$ = \{(1,1), (1,2), (1,3), (1,4), (1,5), (1,6), (2,1), (2,2), (2,3), (2,4), (2,5), (2,6), (3,1), (3,2), (3,3), (3,4), (3,5), (3,6), (4,1), (4,2), (4,3), (4,4), (4,5), (4,6), (5,1), (5,2), (5,3), (5,4), (5,5), (5,6), (6,1), (6,2), (6,3), (6,4), (6,5), (6,6)\}\\

Also,The number of elements in the sample space$(S_y)$ = $6 \times 6 = 36.$

\end{document}
